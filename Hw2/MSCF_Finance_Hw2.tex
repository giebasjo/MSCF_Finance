\documentclass[12pt]{article}
\usepackage{graphicx}
\usepackage{float}
\usepackage{amssymb}
\usepackage{booktabs}

\title{MSCF Finance\\ Homework Set 2}
\author{
Jordan Giebas \\
Daniel Rojas Coy \\ 
Lucas Bahia 
}
\date{\today}

\begin{document}
\maketitle

\section{Question 1: \textit{Consumption and Savings}}

	\subsection{Part (a)}
	
		\begin{center}
 		\begin{tabular}{||c c c||} 
 		\hline
 		 & Today (t=0) &  One Year (t=1) \\ [0.5ex] 
 		\hline \hline
 		Income & 1000 & 0 \\ 
 		\hline
 		Trade  & (237.5) & 250 \\
 		\hline
 		 Consume & 762.5 & 250  \\
 		\hline
		\end{tabular}
		\end{center}
		
	\subsection{Part (b)}
	
		Allow $b$ to represent the number of bonds invested in. At $t=0$, we see the consumption $c_{0} = 1000 - 95b$, and 
		at $t=1$, the consumption $c_{1} = 100b$. Following the Utility formula presented in the handout, we would like to maximize
		the following function. 
		%
		$$ U(c) = log(c_{0}) + 0.98log(c_{1}) $$
		%
		Plugging in $c_{0}$ and $c_{1}$, we must optimize 
		%
		$$ U(b) = log(1000 - 95b) + 0.98log(100b) $$
		% 
		\\		
		%
		We set the derivative of $U(b)$ equal to zero, and solve for $b$.
		%
		$$ 0 = \frac{dU}{db} = \frac{98}{100b} - \frac{95}{1000 - 95b} $$
		%
		We find that $b = 5.21$. 
		
		\subsection{Part (c)}
		
			\begin{center}
 			\begin{tabular}{||c c c||} 
 			\hline
 			 & Today (t=0) &  One Year (t=1) \\ [0.5ex] 
 			\hline \hline
 			Income & 0 & 1052.6 \\ 
 			\hline
 			Trade  & 95$b$ & -100$b$ \\
 			\hline
 			 Consume & 95$b$ & 1052.6 - 100$b$  \\
 			\hline
			\end{tabular}
			\end{center}
			%
			In the same fashion, we would like to find the $b$ that maximizes the utility function but with $c_{0} = 95b$ and $c_{1} = 1052.6 - 100b$.
			%
			$$  0 = \frac{dU}{db} = \frac{1}{b} - \frac{98}{1052.6 - 100b} $$
			% 
			We find that $b = 5.31616$. 
			
		\subsection{Part (d)}
		
			As of 9/15/2017, the current US 1 year T-Bill rate is 1.30\%\. Therefore, assuming we buy 2.5 bonds again, we resolve Part (a) using the current
			price of the bond as $P_{0} = \frac{100}{1.0130} = 98.72$\$. We then have the following,
			\begin{center}
 			\begin{tabular}{||c c c||} 
 			\hline
 			 & Today (t=0) &  One Year (t=1) \\ [0.5ex] 
 			\hline \hline
 			Income & 1000 & 0 \\ 
 			\hline
 			Trade  & (246.8) & 250 \\
 			\hline
 			 Consume & 753.2 & 250  \\
 			\hline
			\end{tabular}
			\end{center}
				
		\subsection{Part (e)}
		
			Assume an individual would like to save some fixed amount of money $x$. If the rate were lower, the individual would need to save more at the current time
			in order to generate $x$ in the future.. On the contrary, if the rate were higher, the individual wouldn't need to save as much in order to generate $x$ in the 
			future. 			
		
\section{Question 2: \textit{Means and Variances}}

	\subsection{Part (a)}
	
		We use the following equations, and the information provided, to determine the expected return, standard devation, Sharpe Ratio, and Beta of the portfolio. 
		%
		$$ \mathbb{E}[r_{p}] = \sum_{i} w_{i} r_{i} $$
		$$ Var(r_{p}) = w_{A}Var(r_{A}) + w_{B}Var(r_{B}) + 2w_{A}w_{B}Cov(r_{A},r_{B}) $$
		$$ Sharpe Ratio = \frac{\mathbb{E}[r_{p}] - r_{f}}{\sigma_{p}} $$
		$$ \beta_{p} = \sum_{i} w_{i} \beta_{i} $$
		%
		We summarize the findings in the table below. 
		%
		\begin{center}
 		\begin{tabular}{||c c||} 
 		\hline
 		 Metric & Numerical Value \\ [0.5ex] 
 		\hline \hline
 		$\mathbb{E}[r_{p}]$ & 0.0728 \\ 
 		\hline
 		$Var(r_{p})$ & 0.34176 \\
 		\hline
 		$Sharpe Ratio$ & 0.1534494  \\
 		\hline
 		$\beta_{p}$ & 0.88  \\
 		\hline
		\end{tabular}
		\end{center}

	\subsection{Part (b)}
	
		CAPM establishes a linear relationships between the expected return on the asset and the expected excess return of the market as follows
		%
		$$ \mathbb{E}[r_{i}] = r_{f} + \beta_{i}(\mathbb{E}[r_{m}] - r_{f}) $$
		%
		Let's check that CAPM holds for the assets A and B.\\
		%
		\begin{center}
		Stock A: $0.0680 \stackrel{?}{=} 0.02 + 0.80(0.08-0.02) = 0.0680$ \\ 
		Stock B: $0.1040 \stackrel{?}{=} 0.02 + 1.40(0.08-0.02) = 0.1040$
		\end{center}
		%
		So as we can see, the CAPM paradigm holds for these assets under the information provided. 
		
	\subsection{Part (c)}
	
		Following a similar procedure from Homework 1, we determined the weights for the optimal market portolio. We then establish the Capital Market Line. 
		Using the CML, we then solve for the weights by calibrating to the expected return of the current portfolio. The optimal weights and the metrics of 
		interest (Expected Return, Standard Deviation, Sharpe Ratio, and Beta) are summarized in the tables below.
		%
		\begin{center}
 		\begin{tabular}{||c c||} 
 		\hline
 		$w_{A}$ & 0.768	 \\ 
 		\hline
 		$w_{B}$ & 0.189\\
 		\hline
 		$w_{r_{f}}$ & 0.041  \\
 		\hline
		\end{tabular}
		\end{center}
		%
		\begin{center}
 		\begin{tabular}{||c c||} 
 		\hline
 		 Metric & Numerical Value \\ [0.5ex] 
 		\hline \hline
 		$\mathbb{E}[r_{p}]$ & 0.0728 \\ 
 		\hline
 		$Var(r_{p})$ & 0.2759 \\
 		\hline
 		$Sharpe Ratio$ & 0.1914  \\
 		\hline
 		$\beta_{p}$ & 0.88  \\
 		\hline
		\end{tabular}
		\end{center}
		
	\subsection{Part (d)}
	
		We would suggest the client to buy at least some of the asset if it is uncorrelated with the market, since the diversification will probably increase the risk adjusted returns.
		The maximum price should be a price that have the returns higher than the risk free rate. Since it will be above the risk free rate and its returns are uncorrelated with the other 					assets adding some of this asset will improve the efficient frontier. Specifically,
		%
		$$ \frac{157.5}{P_{0}} = (1+ 0.02) \Rightarrow P_{0} = 154.41 $$
		
	\subsection{Part (e)}
		
		Using the definition of the Sharpe Ratio, this is straightforward with $P_{0} = 150$
		%
		$$ Sharpe Ratio = \frac{\mathbb{E}[r_{p}] - r_{f}}{\sigma_{p}} = \frac{(\frac{157.5}{150}) - 1}{23.848} = 0.001257967  $$
		%
	
	\subsection{Part (f)}
	
		Since the Lumber prices are correlated this means that we can generate the returns combining market assets. Therefore, we would at least need the return defined by 
		the Efficient Frontier of the portfolio of assets A and B. We can use the solver to find the weights of assets that give the $\sigma = 23.848$
		with the weights we can find the return needed. We found the expected return on the Efficient Frontier corresponding to when $\sigma = 23.848$ to be 
		$\mu = 1.0772$. With this return in mind, much like in Part (d), we determine the initial price that will generate this return. We find that $P_{0} = \$75.823$. 
		
		
\section{Question 3: \textit{CAPM}}

	\subsection{Part (a)}
		
		For each of the industries
		%
		\begin{center}
			$[Aero, Guns, Steel, Ships, Beer, Toys, Fin, Rtail]$
		\end{center}
		%
		We must run the following regression,
		$$ r_{i,t} - r_{f} = \alpha_{i} + \beta_{i}(r_{m,t} - r_{f}) + \epsilon_{i} $$
		%
		We use the data analysis feature in Microsoft Excel to run the regressions, and we summarize our findings in the following table. 
		%
		\begin{center}
 		\begin{tabular}{||c c c||} 
 		\hline
 		 Industry &  $\beta_{i,m}$ & $\alpha_{i}$ \\ [0.5ex] 
 		\hline \hline
 		$Aero$ & $1.11339631$ & $0.002110029$  \\ 
 		\hline
 		$Guns$ & $0.761863487$ & $0.004183162$ \\
 		\hline
 		$Steel$ & $1.406022569$  & $-0.006847625$  \\
 		\hline
 		$Ships$ & $1.1128971$ & $-0.001077632$  \\
 		\hline
 		$Beer$ & $0.70326213$ & $0.004168411$  \\
 		\hline
 		$Toys$ & $1.117458206$ & $-0.002335538$ \\
 		\hline
 		$Fin$ & $1.25111787$ & $-0.000151558$ \\
 		\hline
 		$Rtail$ & $0.981050605$ & $0.001276015$ \\ 
 		\hline
		\end{tabular}
		\end{center}
		
	\subsection{Part(b)}
	
		If CAPM is correct, the $\alpha_{i}$ term should be zero for each of the industries, because if there is alpha then, by the equilibrium argument, the 
		demand would increase, the supply would therefore drop, and the price would rise such that it eliminates the excess return or the alpha. In each of the 
		regressed equations in the table above, the alpha values are all on the order of magnitude $10^{-3}$ or $10^{-4}$, approximately zero. Therefore, 
		the CAPM paradigm fits the data well. 
		%
	\subsection{Part (c)}
	
		The following table ranks the industry be descending $\beta_{i,m}$. 
		%
		\begin{center}
 		\begin{tabular}{||c c||} 
 		\hline
 		 Industry &  $\beta_{i,m}$ \\ [0.5ex] 
 		\hline \hline
 		$Steel$ & $1.406022569$  \\ 
 		\hline
 		$Fin$ & $1.25111787$ \\
 		\hline
 		$Toys$ & $1.117458206$  \\
 		\hline
 		$Aero$ & $1.11339631$  \\
 		\hline
 		$Ships$ & $1.1128971$  \\
 		\hline
 		$Rtail$ & $0.981050605$ \\
 		\hline
 		$Guns$ & $0.761863487$ \\
 		\hline
 		$Beer$ & $0.70326213$ \\ 
 		\hline
		\end{tabular}
		\end{center}
	
	
\section{Question 4: Fama-French Three Factor Model}

	\subsection{Part (a)}
	
		For each of the industries we've been working with, and the data provided, we perform the following regression and summarize the results
		in the following table. 
		%
		$$ r_{i,t} - r_{f} = \alpha_{i} + \beta_{i,m}(r_{m,t} - r_{f}) + \beta_{i,smb}(r_{smb,t} - r_{f}) + \beta_{i,hml}(r_{hml,t} - r_{f}) + \epsilon_{i} $$			
		%
		\begin{table}[H]
 		 \centering
  		  \begin{tabular}{c|cccc}
   		       & $\alpha$ & $\beta_{Market}$ & $\beta_{SMB}$ & $\beta_{HML}$ \\
  		 		 \midrule
  		  		Aero  & 0.001 & 1.164 & -0.018 & 0.307 \\
    			Guns  & 0.003 & 0.818 & 0.050 & 0.417 \\
    			Steel & -0.008 & 1.372 & 0.472 & 0.313 \\
    			Ships & -0.003 & 1.151 & 0.214 & 0.488 \\
    			Beer  & 0.004 & 0.750 & -0.209 & 0.065 \\
    			Toys  & -0.004 & 1.092 & 0.348 & 0.233 \\
    			Finance & -0.001 & 1.269 & 0.106 & 0.234 \\
  		   	    Retail & 0.001 & 0.980 & 0.047 & 0.050 \\
  			  \end{tabular}%
 			\label{tab:addlabel}%
			\end{table}%

	\subsection{Part (b)}
	
		Again, for similar reasons previously discussed, the $\alpha$ values should be zero, which is quite consistent with our multivariate regression results. 
		Nearly all of the $\alpha$ values that we see here are on the order of $10^{-3}$, which is approximately zero.
		
	\subsection{Part (c)}
	
		Although there are multiple $\beta$ factors here, to establish an ordering of riskiness, we summed the coefficients of all the factors for each of the 
		industry and then reported them in descending order. From this, we determine that $Beer$ is the most risky, but $Rtail$ is not far behind. The following
		table summarizes these findings.  	
		%		
		\begin{center}
 		\begin{tabular}{||c c||} 
 		\hline
 		 Industry &  $\sum_{k}\beta_{i,k}$ \\ [0.5ex] 
 		\hline \hline
 		$Beer$ & $0.6085$  \\ 
 		\hline
 		$Rtail$ & $0.5922$ \\
 		\hline
 		$Fin$ & $0.4160$  \\
 		\hline
 		$Aero$ & $0.3726$  \\
 		\hline
 		$Guns$ & $0.2154$  \\
 		\hline
 		$Toys$ & $0.2097$ \\
 		\hline
 		$Ships$ & $0.1327$ \\
 		\hline
 		$Steel$ & $0.1083$ \\ 
 		\hline
		\end{tabular}
		\end{center}
		%
		
\end{document}




