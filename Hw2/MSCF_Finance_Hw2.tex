\documentclass[12pt]{article}
\usepackage{graphicx}
\usepackage{float}
\usepackage{amssymb}

\title{MSCF Finance\\ Homework Set 2}
\author{
Jordan Giebas \\
Daniel Rojas Coy \\ 
Lucas Bahia 
}
\date{\today}

\begin{document}
\maketitle

\section{Question 1: \textit{Consumption and Savings}}

	\subsection{Part (a)}
	
		\begin{center}
 		\begin{tabular}{||c c c||} 
 		\hline
 		 & Today (t=0) &  One Year (t=1) \\ [0.5ex] 
 		\hline \hline
 		Income & 1000 & 0 \\ 
 		\hline
 		Trade  & (237.5) & 250 \\
 		\hline
 		 Consume & 762.5 & 250  \\
 		\hline
		\end{tabular}
		\end{center}
		
	\subsection{Part (b)}
	
		Allow $b$ to represent the number of bonds invested in. At $t=0$, we see the consumption $c_{0} = 1000 - 95b$, and 
		at $t=1$, the consumption $c_{1} = 100b$. Following the Utility formula presented in the handout, we would like to maximize
		the following function. 
		%
		$$ U(c) = log(c_{0}) + 0.98log(c_{1}) $$
		%
		Plugging in $c_{0}$ and $c_{1}$, we must optimize 
		%
		$$ U(b) = log(1000 - 95b) + 0.98log(100b) $$
		% 
		\\		
		%
		We set the derivative of $U(b)$ equal to zero, and solve for $b$.
		%
		$$ 0 = \frac{dU}{db} = \frac{98}{100b} - \frac{95}{1000 - 95b} $$
		%
		We find that $b = 5.21$. 
		
		\subsection{Part (c)}
		
			\begin{center}
 			\begin{tabular}{||c c c||} 
 			\hline
 			 & Today (t=0) &  One Year (t=1) \\ [0.5ex] 
 			\hline \hline
 			Income & 0 & 1052.6 \\ 
 			\hline
 			Trade  & 95$b$ & -100$b$ \\
 			\hline
 			 Consume & 95$b$ & 1052.6 - 100$b$  \\
 			\hline
			\end{tabular}
			\end{center}
			%
			In the same fashion, we would like to find the $b$ that maximizes the utility function but with $c_{0} = 95b$ and $c_{1} = 1052.6 - 100b$.
			%
			$$  0 = \frac{dU}{db} = \frac{1}{b} - \frac{98}{1052.6 - 100b} $$
			% 
			We find that $b = 5.31616$. 
			
		\subsection{Part (d)}
		
			As of 9/15/2017, the current US 1 year T-Bill rate is 1.30\%\. Therefore, assuming we buy 2.5 bonds again, we resolve Part (a) using the current
			price of the bond as $P_{0} = \frac{100}{1.0130} = 98.72$\$. We then have the following,
			\begin{center}
 			\begin{tabular}{||c c c||} 
 			\hline
 			 & Today (t=0) &  One Year (t=1) \\ [0.5ex] 
 			\hline \hline
 			Income & 1000 & 0 \\ 
 			\hline
 			Trade  & (246.8) & 250 \\
 			\hline
 			 Consume & 753.2 & 250  \\
 			\hline
			\end{tabular}
			\end{center}
				
		\subsection{Part (e)}
		
			Something about a convex function here........ write later...
			
		
\section{Question 2: \textit{Means and Variances}}

	\subsection{Part (a)}
	
		We use the following equations, and the information provided, to determine the expected return, standard devation, Sharpe Ratio, and Beta of the portfolio. 
		%
		$$ \mathbb{E}[r_{p}] = \sum_{i} w_{i} r_{i} $$
		$$ Var(r_{p}) = w_{A}Var(r_{A}) + w_{B}Var(r_{B}) + 2w_{A}w_{B}Cov(r_{A},r_{B}) $$
		$$ Sharpe Ratio = \frac{\mathbb{E}[r_{p}] - r_{f}}{\sigma_{p}} $$
		$$ \beta_{p} = \sum_{i} w_{i} \beta_{i} $$
		%
		We summarize the findings in the table below. 
		%
		\begin{center}
 		\begin{tabular}{||c c||} 
 		\hline
 		 Metric & Numerical Value \\ [0.5ex] 
 		\hline \hline
 		$\mathbb{E}[r_{p}]$ & 0.0728 \\ 
 		\hline
 		$Var(r_{p})$ & 0.17088 \\
 		\hline
 		$Sharpe Ratio$ & fdsaf  \\
 		\hline
 		$\beta_{p}$ & fdsaf  \\
 		\hline
		\end{tabular}
		\end{center}

	\subsection{Part (b)}
	
		CAPM establishes a linear relationships between the expected return on the asset and the expected excess return of the market as follows
		%
		$$ \mathbb{E}[r_{i}] = r_{f} + \beta_{i}(\mathbb{E}[r_{m}] - r_{f}) $$
		%
		Let's check that CAPM holds for the assets A and B.\\
		%
		\begin{center}
		Stock A: $0.0680 \stackrel{?}{=} 0.02 + 0.80(0.08-0.02) = 0.0680$ \\ 
		Stock B: $0.1040 \stackrel{?}{=} 0.02 + 1.40(0.08-0.02) = 0.1040$
		\end{center}
		%
		So as we can see, the CAPM paradigm holds for these assets under the information provided. 
		
\section{Question 3: \textit{CAPM}}

	\subsection{Part (a)}
		
		For each of the industries
		%
		\begin{center}
			$[Aero, Guns, Steel, Ships, Beer, Toys, Fin, Rtail]$
		\end{center}
		%
		We must run the following regression,
		$$ r_{i,t} - r_{f} = \alpha{i} + \beta_{i}(r_{m,t} - r_{f}) + \epsilon_{i} $$
		%
		We use the data analysis feature in Microsoft Excel to run the regressions, and we summarize our findings in the following table. For convenience,
		allow $y = r_{i,t} - r_{f}$ and $x = r_{m,t} - r_{f}$. 
		%
		\begin{center}
 		\begin{tabular}{||c c||} 
 		\hline
 		 Industry &  Regressed Equation \\ [0.5ex] 
 		\hline \hline
 		$Aero$ & $y = 1.11339631x + 0.002110029$  \\ 
 		\hline
 		$Guns$ & $y = 0.761863487x + 0.004183162$ \\
 		\hline
 		$Steel$ & $y = 1.406022569x - 0.006847625$  \\
 		\hline
 		$Ships$ & $y = 1.1128971x - 0.001077632$  \\
 		\hline
 		$Beer$ & $y = 0.70326213x + 0.004168411$  \\
 		\hline
 		$Toys$ & $y = 1.117458206 - 0.002335538$ \\
 		\hline
 		$Fin$ & $y = 1.25111787x - 0.000151558$ \\
 		\hline
 		$Rtail$ & $y = 0.981050605x + 0.001276015$ \\ 
 		\hline
		\end{tabular}
		\end{center}
		
	\subsection{Part(b)}
	
		If CAPM is correct, the $\alpha_{i}$ term should be zero for each of the industries, because if there is alpha then, by the equilibrium argument, the 
		demand would increase, the supply would therefore drop, and the price would rise such that it eliminates the excess return or the alpha. In each of the 
		regressed equations in the table above, the alpha values are all on the order of magnitude $10^{-3}$ or $10^{-4}$, approximately zero. Therefore, 
		the CAPM paradigm fits the data well. 
		\newpage
		%
	\subsection{Part (c)}
	
		The following table ranks the industry be descending $\beta$. 
		%
		\begin{center}
 		\begin{tabular}{||c c||} 
 		\hline
 		 Industry &  Regressed Equation \\ [0.5ex] 
 		\hline \hline
 		$Steel$ & $1.406022569$  \\ 
 		\hline
 		$Fin$ & $1.25111787$ \\
 		\hline
 		$Toys$ & $1.117458206$  \\
 		\hline
 		$Aero$ & $1.11339631$  \\
 		\hline
 		$Ships$ & $1.1128971$  \\
 		\hline
 		$Rtail$ & $0.981050605$ \\
 		\hline
 		$Guns$ & $0.761863487$ \\
 		\hline
 		$Beer$ & $0.70326213$ \\ 
 		\hline
		\end{tabular}
		\end{center}
		 
		
			
	
	
\end{document}




